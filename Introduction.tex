\ConfigMan{

\normalsize
\textbf{Document title:} Introduction\\
\textbf{Version:}\footnote{Version number v.X.YY where X=1 "not finished", X=2 "under review" X=3 "finished". YY is incremented each time a change is made, and a short comment about the change is added to the changelog}
 v2.01 \\

\textbf{Github link:} \href{https://github.itu.dk/renha/strEAT/blob/master/MainDocument/sections/Introduction.tex}{.../MainDocument/sections/Introduction.tex} \\
\textbf{Trello card link:} N/A \\

\textbf{Responsible:} ALL\\
\textbf{Status:} Under review\\

\textbf{Changelog:}
\ConRule
%%  DD/MM/YY USER:
\begin{tabular}{@{\noindent}lll}
??/??/??	&?		&Created \\
18/11/18	&renha	&Added configuration management \\
\\
\end{tabular}

\textbf{Comments:}
\ConRule
%%  USERNAME :
\begin{tabular}{@{\noindent}ll}
abru	&Legacy section, imported from Google Docs to GitHub
\\
\end{tabular}
}


\section{Introduction}
In this portfolio, we will go through some of the basic parts of the organisation and management involved in software engineering.
Our goal is to develop a mobile application, and in order to do this, we need to find the best possible way of managing this process as a team.
This portfolio will therefore cover the basic project management of our project.
Chronologically, the first part will be dealing with a proper set up - dividing the roles and responsibilities in the group,
finding the right project management tools, structuring the framework for our work and developing a concrete idea that the whole group can agree on.
We will be exploring the management framework and thereby developing a plan, including more substantial areas like a project kick-off, a case exploration,
research through interviews and, following up, a mock-up version of the app, to be tested by other students.
The tests will afterwards be analysed and the product outlines adjusted accordingly. For clarification and insight, a use case diagram will be provided alongside
dynamic test specification, a configuration management documentation, an exploration of software qualities, a specific quality plan and a class diagram.
Finally, we will thoroughly reflect on the process and the obstacles we are imminently going to come across as a group.
A conclusion will be drawn on both, our experience with the project and our collaboration as a group.
