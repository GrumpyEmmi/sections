%%  dont use real names!!

%% add transcription from workshop max 1-2 pages

%% fololowup on the design from the workshop - add in post workshop reflections.

%% documentation and user stories is the same. - that is documentation


%% focus on just finishing it
% - We can press the "offer discount" button if it will take some time to prepare the food.


% We MAYBE ADD THIS WHOLE REAL APP THING LATER



%\subsection{Native mock-up}

%\begin{figure}
%  \centering
%  \includegraphics[width=\figsize\textwidth]{figs/androidstudio}
%  \caption{Screenshot from android studio which we use to make the native mock-up}
%  \label{guideline for boundaries}
%\end{figure}


%Take pictures and upload them to your group shared space.

%Revise scenarios and class diagrams integrating the changes implied by the mock-up workshop.


%\subsubsection{Guidelines}

%Improvements: Develop common guidelines for the interface: UI pattern, metaphors, and so on.

%\begin{itemize}
%    \item guidelines for theme and colors
%    \item guidelines for logos. See Slack for examples.
%    \item something from androisd studio about how to set delimiters and boundaries.
%\end{itemize}
\ConfigMan{

\normalsize
\textbf{Document title:} Interface Design Using mock-ups\\
\textbf{Version:}\footnote{Version number v.X.YY where X=1 "not finished", X=2 "under review" X=3 "finished". YY is incremented each time a change is made, and a short comment about the change is added to the changelog}
 v2.07 \\

\textbf{Github link:} \href{https://github.itu.dk/renha/strEAT/blob/master/MainDocument/sections/mock-up.tex}{.../MainDocument/sections/mock-up.tex} \\
\textbf{Trello card link:} \href{https://trello.com/c/bkytqqhH/5-mock-up}{trello.com/c/bkytqqhH/5-mock-up} \\

\textbf{Responsible:} renha\\
\textbf{Status:} Under review\\

\textbf{Changelog:}
\ConRule
%%  DD/MM/YY USER:
\begin{tabular}{@{\noindent}lll}
??/??/??	&?		&Created \\
25/10/18	&renha	&Added sub subsections \& figure "Guideline..." \\
26/10/18	&renha	&Added pen and paper mock-up \\
			&		&Updated formatting, \& text \\
29/10/18	&leba	&Updated text \\
31/10/18	&renha	&Added "Implementation" section \\
			&		&Updated formatting \& text \\
01/11/18	&renha	&Added pictures from the mock-up workshop \\
02/11/18	&renha	&Added revised mock-up \\
			&		&Updated formatting \& text \\
18/11/18	&renha	&Added configuration management \\
25/11/18  &leba  &Consistency checks \\
\\
\end{tabular}

\textbf{Comments:}
\ConRule
%%  USERNAME :
\begin{tabular}{@{\noindent}ll}
renha	&This section needs to cleaned up a bit
\\
\end{tabular}
}


\section{Interface Design Using Mock-Ups}
In this section we will:

\begin{itemize}
\item Identify functionality and design our software 'through the interface' together with the user.
\item Gain an understanding of the functionality of the system and the necessary information to support development of mock-ups.
\item Arrange a mock-up workshop with the user evaluation and usability tests.
\end{itemize}

To produce some useful mock-ups for the mock-up workshop we decided to iterate the itterate the Mock Ups in three steps:

\begin{enumerate}
    \item \textbf{Pen and paper mock-up:} Quck sketchup on paper of the app functionality
    \item \textbf{Digial mock-up:} Higher quality mock-ups using image editing software
    \item \textbf{Native mock-up:} implementatino of the mock-ups on their intended target platforms.\footnote{In our case Android and iOS.}
\end{enumerate}


\subsection{Senarios}
We choose to develop the mock-ups, and showcase our app, from the User and Vendor side respectively. We agrees to produce six images showing the following functionality: \\
\textbf{User Side:}
\begin{enumerate}
    \item \textbf{Login screen:} How the login screen will look for the User
    \item \textbf{Main menu:} How the main menu will look for the User.
    \item \textbf{Filtering:} How the user can apply filtering on different parameters in searches
    \item \textbf{Maps:} How the map and location feature will look for the User
    \item \textbf{Menus:} How the User will interact the Vendor menus:
    \item \textbf{Skip the line:} How the "Skip the line" feature will look for the User
\end{enumerate}
\textbf{Vendor Side:}
\begin{enumerate}
    \item \textbf{Login screen:} How the login screen will look for the Vendor
    \item \textbf{Main menu:} How the main menu will look for the User.
    \item \textbf{Maps:} How the Vendor can use the maps integration.
    \item \textbf{Edit menu:} Hot the Vendor can edit his information.
    \item \textbf{Orders:} How the orders from the "Skip the line" feature will look for the Vendor.
    \item \textbf{Businesstat:} Overview of how the Vendor can use \textit{strEAT} to improve her business.
\end{enumerate}

\subsection{Pen and Paper Mock-Ups}


\begin{figure}
   \centering
   \begin{subfigure}[b]{0.475\textwidth}
       \centering
       \includegraphics[width=\textwidth]{figs/penuser}
       \caption{The pen and paper mock-ups for the Vendor side}
       \label{penvendor}
   \end{subfigure}
   \hfill
   \begin{subfigure}[b]{0.475\textwidth}
       \centering
       \includegraphics[width=\textwidth]{figs/penvendor}
       \caption{The pen and paper mock-ups for the User side}
       \label{penuser}
   \end{subfigure}
\end{figure}

\subsection{The Mock-Up Workshop}
In order to recieve feedback on our initial Pen and paper mock-ups, we arranged a mock-up workshop.
In this section we will discuss the preperation and implementation of the workshop. We interviewed two students for the mock-up workshop. We let them try out our mock-ups and got the Following feedback:\\
\textbf{Mikkel SDT:}\\
Mikkel had the following comments about out mock-up:

\begin{itemize}
    \item Lacking back buttons
    \item Turn into swipe
    \item I would like to be tempted by places. So just the finding is the nice function AND spotting the menu, NOT necessarily order it.
    \item Vendor point of view seems nice
    \item Skipping the line also might try
\end{itemize}
\textbf{Jonathan:}\\
Jonathan also had some comments along with further recomendations:

\begin{itemize}
    \item Not enough filters (let’s add them)
    \item “I want halal”
    \item (“I don’t want kosher”  )
    \item More cryptocurrency options (bitConnect)
    \item Think already exist
    \item You want special offers, discount when using an app.
    \item With the skip line you cannot have recommendations but it is nice that.
    \item I know what to expect from menu descriptions and photos of food.
\end{itemize}


\begin{figure}[h!]
    \centering
    \begin{subfigure}[b]{0.475\textwidth}
        \centering
        \includegraphics[width=\textwidth]{figs/mockup}
    \end{subfigure}
    \hfill
    \begin{subfigure}[b]{0.475\textwidth}
        \centering
        \includegraphics[width=\textwidth]{figs/mockup2}
    \end{subfigure}
    \label{mockup1}
    \caption{Pictures from our mock-up workshop.}
\end{figure}



%\subsection{Post workshop reflections.}
%From the workshop we got a ot of good ideas and important input for our revised mock-ups




\subsubsection{Revised Mock-Up.}
With the feedback from the mock-up we were able to make a revised set of Mock Ups foor the user and vendorside respectively.

\begin{figure}[h!]
    \centering
    \begin{subfigure}[b]{0.475\textwidth}
        \centering
        \includegraphics[width=\textwidth]{figs/UserMockupRevised}
    \end{subfigure}
    \hfill
    \begin{subfigure}[b]{0.475\textwidth}
        \centering
        \includegraphics[width=\textwidth]{figs/UserMockupRevised}
    \end{subfigure}
    \label{MockupRevised}
    \caption{Our revised mock-up}
\end{figure}

\subsubsection{Logo and Colours - Look and Feel}
We had a session where we tried out different ideas for logos. you can see the results of this brainstorm in fig.\ref{fig:logos_all}

\begin{figure}
       \centering
       \begin{subfigure}[b]{0.475\textwidth}
           \centering
           \includegraphics[width=\textwidth]{figs/logos/logo.png}
           \caption{Logo 1}
           \label{fig:logo1}
       \end{subfigure}
       \hfill
       \begin{subfigure}[b]{0.475\textwidth}
           \centering
           \includegraphics[width=\textwidth]{figs/logos/logoemmi}
           \caption{Logo 2}
           \label{fig:logo2}
       \end{subfigure}
       \vskip\baselineskip
       \begin{subfigure}[b]{0.475\textwidth}
           \centering
           \includegraphics[width=\textwidth]{figs/logos/logo2.png}
          \caption{Logo 3}
           \label{fig:logo3}
       \end{subfigure}
       \quad
       \begin{subfigure}[b]{0.475\textwidth}
           \centering
           \includegraphics[width=\textwidth]{figs/logos/logo4.png}
           \caption{Logo 4}
           \label{fig:logo4}
       \end{subfigure}
       \caption{Logo Proposals}
       \label{fig:logos_all}
   \end{figure}



   %\subsection{Digital mock-up}

   %\begin{itemize}
%       \item COMMING LATER - MAYBE
%   \end{itemize}
