\ConfigMan{

\normalsize
\textbf{Document title:} Scrum\\
\textbf{Version:}\footnote{Version number v.X.YY where X=1 "not finished", X=2 "under review" X=3 "finished". YY is incremented each time a change is made, and a short comment about the change is added to the changelog}
 v2.02 \\

\textbf{Github link:} \href{https://github.itu.dk/renha/strEAT/blob/master/MainDocument/sections/Scrum.tex}{.../MainDocument/sections/Scrum.tex} \\
\textbf{Trello card link:} N/A \\

\textbf{Responsible:} leba\\
\textbf{Status:} Under review\\

\textbf{Changelog:}
\ConRule
%%  DD/MM/YY USER:
\begin{tabular}{@{\noindent}lll}
18/11/18	&leba	&Created \\
18/11/18	&renha	&Added configuration management \\
23/11/18	&leba	&Added reference to figure \\
25/11/18  &leba &Figure removed and consistency checks \\
25/11/18  &leba &Removed introduction heading \\
\\
\end{tabular}

\textbf{Comments:}
\ConRule
%%  USERNAME :
\begin{tabular}{@{\noindent}ll}

\\
\end{tabular}
}

\section{Scrum}
%% Explain the ideology and general outline of the framework
Scrum is one of the several frameworks that fall under the agile development umbrella. Today, it is one of the most popular frameworks, partially because of the difficulties it addresses in relation to the management of agile software development, since it was developed as a method to wisely manage and secure the time and resources when applying agile software processes \cite{Sommerville}. As a project management model with a high level of user interaction and a “…condensed time-to-market”, Scrum has become highly popular among mobile companies and application developers, as it provides a good combination of functions such as project overview, to-do’s and task lists \cite{Flora}.
\\
Scrum consists of several artifacts, roles and events \cite{Schwaber}. These will be discussed briefly below and in detail in the Roles and Responsibilities, and Planning and Estimation sections later in the report.

\subsection{Scrum Artifacts}
\subsubsection{Product Backlog}
The Product Backlog essentially lists the requirements of the project. The items may take the form of user stories, formal/informal requirements, required knowledge acquisition, etc. The list is subject to change throughout the life of the project.
\subsubsection{Sprint Backlog}
A subset of the Product Backlog, the Sprint Backlog consists of items to be completed during a Sprint, an event which will be introduced below. The Development Team completes the tasks of the Sprint Backlog to produce the next viable product iteration.
\subsubsection{Iteration}
The Iteration is simply a version of the deliverable, updated through the work completed during each Sprint cycle.

\subsection{Scrum Team}
\subsubsection{Product Owner}
The Product Owner represents the interest of the stakeholders and is responsible for overseeing the quality of output from the Development Team. They are also responsible for managing the Product Backlog.
\subsubsection{Scrum Master}
The Scrum Master supports the Development Team by ensuring all members understand the principles of Scrum and its implementation. They help manage the Product Backlog, along with the Product Owner, and liase with the Development Team with regards to its formation. They also facilitate any Scrum event.
\subsubsection{Development Team}
The Development Team are responsible for completing the tasks defined in each Sprint Backlog. They are involved in discussions regarding the formation of the Product and Sprint Backlogs.

\subsection{Scrum Events}
\subsubsection{Sprint}
A series of successive Sprints make up the project timeframe. The length of a Sprint will vary between projects but typically ranges from 2-4 weeks. During each Sprint, items from the Sprint Backlog are completed, with the outcome a new iteration of the product to be delivered.
\subsubsection{Sprint Planning}
As the name suggests, this event involves planning for an upcoming Sprint. This requires populating a Sprint Backlog, which may require deliberation between team members. The Development Team then have to determine how the work set out by the Sprint Backlog will be completed and who will do each part. It may be helpful to also outline a goal for the Sprint.
\subsubsection{Daily Scrum}
The Daily Scrum, or Stand-Up, is an short meeting that takes place at the same time every day and allows the Development Team to state what will be done in the next 24-hour period.
\subsubsection{Sprint Review}
This is an informal post-Sprint meeting between the Scrum Team and stakeholders to inspect the increment and propose changes to the Product Backlog if necessary. The Product Owner states what has been done and how the Product Backlog currently stands. The Development Team states how they felt the Sprint went, including problems faced and solutions used. All parties discuss what to do in the next Sprint.
\subsubsection{Sprint Retrospective}
This meeting includes only the Sprint Team and does not concern the stakeholders. It is intended to discuss the "... people, processes, relationships, and tools" of the Sprint. The Sprint Team may discuss ways to improve their own performance in upcoming Sprints.

\subsection{Conclusion}
The sections to follow in this report will refer to, and elaborate on, the artifacts, roles, and events briefly outlined above. In particular, the section entitled Roles and Responsibilities will delve a little deeper into the aspects of the Scrum Team and assign roles to each team member, the Planning and Estimation section will constuct Product and Sprint Backlogs for the course deliverables themselves, and the Requirements section will do the same with regards to \textit{strEAT} product requirements.
