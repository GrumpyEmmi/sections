\ConfigMan{

\normalsize
\textbf{Document title:} Roles and Responsibilities\\
\textbf{Version:}\footnote{Version number v.X.YY where X=1 "not finished", X=2 "under review" X=3 "finished". YY is incremented each time a change is made, and a short comment about the change is added to the changelog}
 v3.05 \\

\textbf{Github link:} \href{https://github.itu.dk/renha/strEAT/blob/master/MainDocument/sections/Roles.tex}{.../MainDocument/sections/Roles.tex} \\
\textbf{Trello card link:} \href{https://trello.com/c/ysioskYb/14-roles-and-responsibilities}{trello.com/c/ysioskYb/14-roles-and-responsibilities} \\

\textbf{Responsible:} jopo\\
\textbf{Status:} Finished\\

\textbf{Changelog:}
\ConRule
%%  DD/MM/YY USER:
\begin{tabular}{@{\noindent}lll}
13/09/18	&jopo	&Created \\
01/11/18	&abru	&Added Latex tables \\
02/11/28	&leba	&Updated text \\
18/11/18	&leba	&Updated text \\
18/11/18	&renha	&Added configuration management \\
23/11/18	&leba	&Updated formatting \& text \\
\\
\end{tabular}

\textbf{Comments:}
\ConRule
%%  USERNAME :
\begin{tabular}{@{\noindent}ll}
abru	&Legacy section, imported from Google Docs to GitHub
\\
\end{tabular}
}

\section{Roles and Responsibilities}

This section outlines the roles and responsibilities delegated to each team member as per the aforementioned Scrum framework. Scrum emphasizes interactions between individuals and the terminology for roles is unlike most other project management systems. The Sprint Cycle section will outline how we will conduct each Sprint and it's events. The Roles section continues where the Scrum section left-off by formally stating the expectations of the Product Owner, Scrum Master and Development Team. The final section assigns roles to our team members and states their responsibilities with regards to \textit{strEAT}.

\subsection{Sprint Cycle}

During the Scrum sprint, all team members share information, describe their progress since the last meeting, bring up problems that have arisen, and state what is planned for the following day. Thus, everyone on the team knows what is going on and, if problems arise, team can re-plan short-term work to cope with them. Everyone participates in this short-term planning; there is no top-down direction from the Scrum Master. Daily scrum meetings, in our case, are going to be conducted in Slack. During weekly meetings, the estimated work remaining in the Sprint will be calculated and graphed by the Scrum Master \cite[p. 87]{Sommerville}. However, we can coordinate interactions between members daily with Trello that will fulfil a role of a Scrum board. In offices, it is usually a whiteboard that includes information and post-it notes about the sprint backlog, work done and work in progress (www.mountaingoatsoftware). In our case, Trello would be a good substitute, as everyone can update it daily and have a constant and easy access to it. This is a shared resource for the whole team, and anyone can change or move items on the board. It means that any team member can, at a glance, see what others are doing and what work remains to be done. We may also create a sprint burndown chart to see a percentage of work finished.

Generally, the tasks should be achievable in less than one day, however, our tasks will be set to one-week capacity – so that we can discuss them on weekly meetings. Once the Sprint Backlog is decided, it is fixed. The team cannot remove items from it (but adding items is possible). If an item isn't finished, the Product Owner will need to decide what to do with it.  The Product Owner is the only individual that can remove things from the Sprint Backlog if it no longer provides business value.

On the final day of each sprint, a review session (Sprint Review meeting) is conducted to allow the Product Owner to check if all of the committed items are complete and implemented correctly. Additionally, a Sprint Retrospective is conducted to check and improve the project execution processes: what was good during the Sprint, what should continue as it is and what should be improved (www.clearlyagileinc). The team will benefit from the Sprint Backlog as it gives direction on a day-to-day basis, keeping the group on track.


\subsection{Scrum Roles and Responsibilities}

To summarize, we have three different roles in the Scrum process: Product Owner, Scrum Master and Scrum Team. Their responsibilities are:

\subsubsection{Product Owner}
\begin{itemize}
\item Ensures that the level of detail in the specification of the backlog items is appropriate for the work to be done.
\item Involved in prioritising the items on the Product Backlog to define which are the most important.
\item On the first day of the new sprint, the Product Owner adds new items to the Product Backlog.
\item Responsible for conveying the vision of the stakeholders to the team.
\item Responsible for the return on investment (ROI).
\item Communicates with the stakeholders about progress and problems.
\item Selects which of the highest priority items they believe can be completed in the upcoming sprint.
\item Acts as a communcation channel between the team and other involved parties (stakeholders).
\end{itemize}

\subsubsection{Scrum Master}
\begin{itemize}
\item Conducts all Scrum ceremonies and processes.
\item Removes all hindrances or disturbances to achieving each goal.
\item Selects which of the highest priority items they believe can be completed in the upcoming spring (together with the rest of the Scrum team and Product Owner).
\item Commits to goals and deadlines on behalf of the team.
\item Does not interfere with the decisions of the team, specifically regarding the development, but rather is there as an advisor for the team.
\end{itemize}

\subsubsection{Development Team}
\begin{itemize}
\item Responsible for all the activities who's completion are required to achieve the sprint goals.
\item Once committed, it is a team's responsibility to fulfil the commitment and deliver the agreed upon results, on time, and with high quality.
\item Team members have to participate in the meetings and have to ensure that all the findings of the meetings are being addressed in the project (www.knowledgehut.com)
\item Involved in the selection of the highest priority items they believe can be completed in the upcoming sprint (together with Scrum Master and Product Owner).
\end{itemize}
%what if i push and then ask me for pull - dnt i loose my changes
%hyperlink

\subsection{Scrum Roles and Responsibilities in our Project}
Our team has decided that in order to achieve fairness we will let everyone try different roles in Scrum. Also, we believe rotating roles will enhance collaboration between members and avoid monotony. Hence, below is the plan for Scrum Masters (SM) and Product Owners (PO) over the course of the semester. Week 11-14 will be decided upon later, based on team members experience with fulfilling different functions and their feelings about them.

\begin{table}[h!]
\centering
\begin{tabular}{@{}ccc@{}}
\toprule
\textbf{Week\#} & \textbf{Product Owner} & \textbf{Scrum Master} \\ \midrule
4 (38)          & Anders                 & Emmi                  \\ \midrule
5 (39)          & Ida                    & Jowita                \\ \midrule
6 (40)          & Katrine                & Lewis                 \\ \midrule
7 (41)          & René                   & Anders                \\ \midrule
8 (43)          & Emmi                   & Ida                   \\ \midrule
9 (44)          & Jowita                 & Katrine               \\ \midrule
10 (45)         & Lewis                  & René                  \\ \midrule
11 (46)         & -                      & -                     \\ \midrule
12 (47)         & -                      & -                     \\ \midrule
13 (48)         & -                      & -                     \\ \bottomrule
\end{tabular}
\caption{Our roles for each week of the project}
\label{table_roles}
\end{table}

\subsubsection{Product Owner}

Our Product Owner’s most important responsibility is the adding of new items (with a proper level of specification) to the Trello ‘To Do’ list on the first day of the new sprint. They also need to prioritise those items by moving tabs with the most important tasks to be done (in the current cycle) on the top of the Trello board. Moreover, the PO is a communication channel between Silviu and the team, which includes: conveying Silviu's comments to the team members that were not present during the weekly meeting, and keeping Silviu up to date with regards to progress or problems faced by the team.

\subsubsection{Scrum Master}

Scrum Master makes sure that everyone follows Scrum roles and ceremonies agreed upon at the beginning of the project and makes sure every tool (Trello as a planning instrument and Google Docs for documents sharing) is working properly and is being updated according to the work done.
Moreover, the SM announces the goals and deadlines to Silviu on behalf of the team but without interfering with the decisions of the team regarding the content of the tasks. He or she is in charge of minutes from meetings (both with and without a supervisor) and inserting them on a proper Slack channel. Finally, the Scrum Master will send or upload proper documents for the evaluation to the supervisor and teacher through LearnIt.

\subsubsection{Development Team}
As soon as a team member commits to a task it is their responsibility to fulfil the commitment and deliver the agreed upon results, on time, and with high quality. To ensure this quality, team members are also required to review each other's work via shared Google Docs document. Updating Trello is essential and each team member moves their tasks to the proper boards (from ‘To Do’ to ‘Pending’ and later to ‘Finished’) remembering to keep the prioritised order while doing so. All team members have to participate in the meetings (one absence is acceptable). If unable to be at the meeting in person, Skype attendance is required. For critical situations that require a fast answer, the team uses the appropriate channel on Slack. All team members that are called on the channel are required to help and give feedback. THe team must be available on Slack throughout all weekdays.

All of us take part in the discussion regarding which of the highest priority items can be completed and what is the time required to complete said items. Also, the whole team is responsible for the proper use of Slack channels and each member is an active part of them.
