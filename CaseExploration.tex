\ConfigMan{

\normalsize
\textbf{Document title:} Case Explorations\\
\textbf{Version:}\footnote{Version number v.X.YY where X=1 "not finished", X=2 "under review" X=3 "finished". YY is incremented each time a change is made, and a short comment about the change is added to the changelog}
 v3.14 \\

\textbf{Github link:} \href{https://github.itu.dk/renha/strEAT/blob/master/MainDocument/sections/CaseExploration.tex}{.../MainDocument/sections/CaseExploration.tex} \\
\textbf{Trello card link:} \href{https://trello.com/c/ZuWPvGHd/10-explore-your-case}{trello.com/c/ZuWPvGHd/10-explore-your-case} \\

\textbf{Responsible:} ALL\\
\textbf{Status:} Finished\\

\textbf{Changelog:}
\ConRule
%%  DD/MM/YY USER:
\begin{tabular}{@{\noindent}lll}
??/??/??	&?		&Created \\
09/09/18	&mgab	&Added Marie-Louise's case exploration \\
09/09/18 	&jopo	&Added Jowita's case exploration \\
09/09/18 	&idbo 	&Added Ida's case exploration \\
10/09/18 	&renha	&Added René's case exploration \\
10/09/18 	&kati 	&Added Katrine's case exploration \\
10/09/18 	&leba	&Added Lewis' case exploration \\
12/09/18 	&leba 	&Added references to Lewis' section \\
13/09/18 	&abru 	&Added Anders' case exploration \\
13/09/18 	&kati 	&Added references to Katrine's section \\
13/09/18 	&renha	&Added references to René's section \\
14/09/18 	&renha	&Updated René's case exploration \\
02/11/18	&leba	&Updated grammar \\
18/11/18	&renha	&Added configuration management \\
23/11/18	&leba	&Updated formatting \\
25/11/18  &leba &Case description and formatting \\
\\
\end{tabular}

\textbf{Comments:}
\ConRule
%%  USERNAME :
\begin{tabular}{@{\noindent}ll}
abru	&Legacy section, imported from Google Docs to GitHub \\
\\
\end{tabular}
}

\section{Case Exploration}
In this section, as we have deviated from this course's prescribed cases, we first present a case description. Later, each team member expands on the case description. Each team member has a different perspective on the case and may draw on skills from their former academic background.

\subsection{Case Description}
Street food is getting very popular in Copenhagen - for example, "Papirøen" has expanded and become "Reffen". Also, there are different vendors who move around in the city. At present, there is no platform for communication between street food vendors and hungry customers. This means that vendors are currently losing potential revenue and customers find themselves in a position where they cannot find the street food stall of their preference since they are moving around. Currently, the closest competitors in this market are sites like JustEat and Wolt, but they do not include smaller, non-stationary food companies at present.

\subsection{Katrine}
Street food has become a growing phenomenon in a range of urban centres all over the world. Compared to its roots in the developing world, this practice has consequently moved slowly into the developed parts of the world \cite{Tinker}. In Copenhagen, we’ve had a long tradition of the so-called “Pølsevogn” \footnote{Translates: sausage wagon}, which today is becoming a more extinct type of enterprise due to a change in Danish food culture.  However, others would argue that it is a part of Danish cultural heritage. Today we are experiencing how the street food culture in Copenhagen is expanding rapidly, and also becoming a means for tourism. An additional important aspect to consider, which Irene Tinker \cite{Tinker} points to, is how urban street food has become:
\begin{quotation}
“[…] a critical element in the survival strategies of the urban poor […]”.
\end{quotation}
With this in mind, it seems like an interesting and relevant project to develop the street food app \textit{strEAT}, which will hopefully be a great tool for street food vendors in this micro-enterprise business, as it will allow the customers to have better access to vendors.

\subsection{Ida}
In recent times, the UK has felt the popularity of street food, and its emergence is largely due to the heavy interest from investors \cite{Butler}. Street food has also gained popularity over the past few years in Copenhagen. Copenhagen Street Food opened a street food market in 2014, at Papirøen (Visit Copenhagen), and moved to Refshaleøen in 2018. Street food trucks are also seen in Kødbyen Vesterbro, Copenhagen and Verdenshjørnet Nørrebro, Copenhagen. Some of the food trucks have a permanent spot, while others move around to different spots in Copenhagen. The food trucks without a permanent spot can be hard to find for the customer, and there is neither an app where the customer can find the food truck nor order food from (not even from the food trucks with a permanent spot). Since popularity is rising for street food and investors have found their way into the market, one could argue that there is a need for an app that is specific to street food trucks.

\subsection{Jowita}
Building a community willing to be a part of the app is one of our biggest challenges. Street food vendors are the very core of this community and their hesitation to join it must be taken into consideration. Besides the obvious advantages mobile ordering apps bring to the business, more and more restaurants point out problems that they brought to the operations. It even became a ‘fashion’ for the restaurants to cut ties with mobile ordering apps \cite{wsj}. While restaurant owners admit that food delivery platforms like UberEATS increase the number of orders their businesses receive, a lot of them also say that these apps increase operational headaches behind the scenes and that they don’t have a significant effect on profit margins \cite{eater}. This is especially true when a food vendor juggles between multiple apps and websites, therefore it is vital that we convince street food businesses to focus on this one, tailored directly to their needs, \textit{strEAT} app. To our advantage, street food vendors usually have lower revenue than regular restaurants and thus, unlike them, they don’t have the bandwidth to develop their own apps (restaurants' own ordering platforms are becoming one of the biggest threats for food ordering apps). Nevertheless, in order to convince street food vendors to join us, we need to simplify operations for them as much as possible (for example, by communicating directly with kitchen staff).

\subsection{Marie-Louise}
During the last few decades, food has become one of the most present elements of people’s lifestyles that has reached its peak with social media enabling consumers to display the objects of consumption and the location of such a purchase. The majority of elements of people’s everyday lives have to follow aesthetic-driven marketing strategies that appeal not only to the eye but also to the Instagram feed. Hence, the demand for aesthetically pleasing food, and sustainable concepts behind it, has spread from the high street to smaller cafes, shops and markets. Street food markets are one of the most concise examples of this phenomenon. They’ve come a long way from being mainly frequented by workers in their short breaks, or offering only simple snacks, to now mostly being stylish venues with marketing strategies, investors and sophisticated food options, often catering to the needs of people with allergies or ethical food consumption guidelines. A melting pot for representative examples is the Reffen food market in Copenhagen that includes 70 food containers with innovative concepts. That offers a lot of choice, especially in combination with all the other individual food trucks and bikes in and around the Copenhagen area. As great as that is for the consumer, it is also confusing: there is no way for the (potential) customer to find information regarding not only vendor menus but also prices, locations and sometimes even opening hours.
%
Not even Reffen, as the biggest concept, has any detailed information on their website. It therefore seems overdue for street food vendors to take a step further into modern marketing options and explore economical growth opportunities that come with a virtual network of customers and vendors alike, where information can be retrieved and the aforementioned elements are made more transparent. Delivery services in Copenhagen, like Wolt or JustEat \cite{JustEat}, are currently covering most establishments but are neglecting food market trucks and stalls. Therefore, it seems a natural consequence for that sector of the market to be given a sufficient option to advertise and display their food, even when their potential customers are not on location, and, potentially, also to distribute it to the customer as to be fully able to compete with restaurants.

\subsection{René}
I want to use my time for the "explore my case" to investigate the current market a bit:
Searching on Google, I found this app which does street food \cite{streetfoodapp}, which is in the direction we had in mind, running in Vancouver, Canada. There is also another variation on this idea which is available in Britain from “British Street Food Development”. They write on their website:

"It’s revolutionary, with live GPS maps showing who’s trading where and when. It details the specials of the best traders, and encourages punters to photograph – and review – their food. The idea is that it is helping to build a bigger street food community. And, for the first time this year, it allowed people to vote for their favourite traders at the British Street Food Awards." \cite{britishstreetfood}

The (totally revolutionary) live GPS maps™  is also a core feature of the app that we want to develop, and we have also discussed other modes of income for the company with features like reviews and photo service as possible extensions.  We could also add an online shop which will empower our customers with the option to do catering. Street food is getting pretty cool in Copenhagen and notably we have “Reffen” as a hub for the Copenhagen based market and it will be a good place to pitch our idea for the kick-off. I’ve not been able to find any indication that others are working on this in Copenhagen.


\subsection{Lewis}
An estimated 2.5 billion people eat street food every day \cite{Fellows}. The culture, made popular by developing nations as a means of acquiring cheap food, now seemingly pervades every corner of the globe. In the US alone, the current food truck market, which is valued at USD 856m, is expected to rise to approximately USD 1b by 2020, having sustained an average annual growth of 7.9\% since 2011 \cite{The_Economist}. A figure far surpassing that of its brick-and-mortar counterpart, which stands at 2\% \cite{Galatro}. Not one to be left behind, Europe has also begun to warm up to the idea of a burgeoning street food market. In the UK, vendors can expect to begin plating up for as little £2,000 \cite{Rivera}, a relatively small investment which encourages competition and variety. Staying in Britain, 18-34 year olds are eating out approximately 25\% of the time, well above the national average of 15\% \cite{Stenning}. Of the £6.1b restaurant market, approximately £0.9b is attributed to street food and mobile van revenue, a 14.7\% share. With nearly half of all internet traffic being attributed to mobile devices \cite{Panko}, 91\% of smartphone users using apps [?], and street food consumption trending upward, the success of casual dining applications such as Just Eat, UberEATS, Wolt and Deliveroo are of no surprise. However, these services deal with delivery, and a gap in the market still remains for venues with no fixed location who are looking to attract new clients. This will be addressed by our new application, \textit{strEAT}.

\subsection{Anders}
Food is one of very few “products” that have the amazing property of being indispensable to all people, all with their own tastes and preferences. As such, the market for food is naturally ripe with opportunities by catering to various niches of consumers. However, new opportunities usually lead to new challenges. With our app, we want to solve one of the inherent challenges related to the phenomenon of street food. Most people probably have a favourite place to eat, and while your preferred brick-and-mortar restaurant won’t be going anywhere in a physical sense; your favourite food truck is very likely to. With \textit{strEAT} we want to address this, by enabling consumers to keep track of the street food vendors at all times using existing GPS technology. At the same time, we want to give the vendors a platform for communicating with their customers, and enable them to improve their business through this communication. Additionally, we want to give street food lovers a platform, not only for locating the vendors they know, but also to explore the market of street food and discover new vendors – and hopefully new favourites.
