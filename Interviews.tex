\ConfigMan{

\normalsize
\textbf{Document title:} Appendix\\
\textbf{Version:}\footnote{Version number v.X.YY where X=1 "not finished", X=2 "under review" X=3 "finished". YY is incremented each time a change is made, and a short comment about the change is added to the changelog}
 v3.04 \\

\textbf{Github link:} \href{https://github.itu.dk/renha/strEAT/blob/master/MainDocument/sections/Interviews.tex}{.../MainDocument/sections/Interviews.tex} \\
\textbf{Trello card link:} \href{https://trello.com/c/O4hq8etf/7-explore-the-use-context}{trello.com/c/O4hq8etf/7-explore-the-use-context} \\

\textbf{Responsible:} idbo \& kati\\
\textbf{Status:} Finished\\

\textbf{Changelog:}
\ConRule
%%  DD/MM/YY USER:
\begin{tabular}{@{\noindent}lll}
02/10/18	&idbo		&Created \\
02/10/18	&kati	  &Added Interview Guide \\
02/10/18	&idbo		&Added Interview - Bar manager \& Interview - Student 1 \\
05/10/18	&kati		&Added Interview - Student 2 \\
18/11/18	&renha	&Added configuration management \\
\\
\end{tabular}

\textbf{Comments:}
\ConRule
%%  USERNAME :
\begin{tabular}{@{\noindent}ll}
abru	&Legacy section, imported from Google Docs to GitHub

\\
\end{tabular}
}





\section{Appendix}
\label{Interviews}

\subsection{Interview Guide}
We choose to do semi-structured interviews. This means that we wrote a few basic questions for two different kind of interviews, namely potential customer interviews and potential vendor interviews. Since this kind of interview is not comepletely structured, as its name indicates, we often came with follow-up questions durring the interviews. Below is the list of questions we wrote before conducting the interviews.


\subsubsection{Customers}

\begin{enumerate}
  \item Do you use any apps to buy/find food in Copenhagen?
  \item If so, what are they called?
  \item Why do you use them?
  \item What should a food app contain / be able to do / help you with?
  \item What do you think about an app that could show you where to find street food in Copenhagen?
  \item Would you ever consider using such an app?
  \item Why? Why not?
  \item What should/could this app be able to do if you should use it?
\end{enumerate}


\subsubsection{Street food Vendors}

\begin{enumerate}
  \item Do you always have your stand in the same location?
  \item Do you use social media / apps to advertise your business?
  \item If not, would you be interested / or what do you think of that idea?
  \item What if there existed an app to show the location of street food vendors in Copenhagen, where customers could also order food. Do you think you would use that?
  \item What should this app be able to do if you should use it?
  \item Do you have any additional ideas of how such an app could benefit your business or your work?
\end{enumerate}


\subsection{Bar manager, Tipi Bar Nørrebro:}

\textit{The interview person is informed about the project. - The recording starts after the first question is asked.}

\textbf{Bar manager:}
\begin{quotation}
When we started the bar or the owner started the bar, it was supposed to be a bar in the middle of a lot of street food and it worked for a while, but because [the food trucks] opened and closed whenever they wanted to... They didn’t have opening hours like stores have, it made it difficult when guests came here, because they maybe came here for a specific kind of food truck and that was maybe closed, and there was maybe two [food trucks] opened out of the ten we had here, so people came here and were more disappointed than happy, so then we slowly built the ‘TIPI BAR’ and started getting more guests here and then the food trucks moved out and now we have the food truck that is actually open which is ‘Chop Stick’ [laugh] and they're super good. So the others moved out.. to other places.. and it was hard having them around because you never knew when they were going to be opened or closed, and it was supposed to be a collaboration between the bar and them, but then it ended up being easier for us to build a big bar where people are allowed to bring food from wherever - they can bring home cooked food, have a romantic dinner here where they buy wine in the bar, or people have their birthday party where they can bring their own cakes and everything. So we did that instead of having food trucks around, but now you can bring food from wherever you want.
\end{quotation}

\textbf{Interviewer:}
\begin{quotation}
And that works fine?
\end{quotation}

\textbf{Bar manager:}
\begin{quotation}
  Yeah, it works fine! Some people get disappointed because they remember the food truck area or they have seen commercials for it and then we kind of have to explain the whole thing again - that it worked during the summer months, but the rest of the year.. and we have a bar and the rent, so unfortunately they had to move, and I don’t know where they moved to.
\end{quotation}

\textbf{Interviewer:}
\begin{quotation}
We were just talking about it, that maybe they are all in ‘Reffen’, maybe they are trying to collect it together.
\end{quotation}

\textbf{Bar manager:}
\begin{quotation}
    I wish they would and that they would actually collaborate and have this ‘Let’s have these opening hours and get people here’, because I like the idea that this place had in the beginning, that it was a food market and people would come here for lunch and they could get so many different kinds of food. So I hope that works, because it would actually be nice to go to a place with a food market, where you could be like ‘what do I feel like eating today?’.
\end{quotation}

\textbf{Interviewer:}
\begin{quotation}
    We were actually thinking our idea was to have this mobile app where you could… For us we would get in contact with the street food vendors and then trying to map them all in Copenhagen, so people would be able to go in and look for what kind of food or what kind of.. price range and find them on a map or list and see where they could find them in Copenhagen and their opening hours, if they have some off course [laugh] and to see the menu or others recommendations about the place. Do you think that would be helpful?
\end{quotation}

\textbf{Bar manager:}
\begin{quotation}
 That’s a very good idea! I don’t know with the Asian one here, whether it’s going to stay, but the other one next to it, the one with falafel, it's not going to stay here because there's too much competition here in Nørrebro when it comes to falafel. He wants his own place where people can actually sit; he wants a inside area.
\end{quotation}

\textbf{Interviewer:}
\begin{quotation}
 That's also a problem if they become too popular, I guess.
\end{quotation}

\textbf{Bar manager:}
\begin{quotation}
    Exactly! So he's doing that, and then we'll only have ‘Chop Stick’ here.
\end{quotation}

\textbf{Interviewer:}
\begin{quotation}
    I don’t think we have any more questions, but thank you for giving us an overview of what happened with the food trucks and for your time.
\end{quotation}


\subsection{Student 1:}

\textit{The interview person is informed about the project. - The recording starts after the first question is asked.}

\textbf{Interviewer:}
\begin{quotation}
    Do you use any apps to buy/find food in Copenhagen?
\end{quotation}

\textbf{Student 1:}
\begin{quotation}
    I use ‘JustEat’ and ‘Wolt’ and ‘ToGoodToGo’.
\end{quotation}

\textbf{Interviewer:}
\begin{quotation}
    Why do you use them?
\end{quotation}

\textbf{Student 1:}
\begin{quotation}
    ‘JustEat’ because of pizza, and ‘Wolt’ because of butterchicken and ‘ToGoodToGo’ because I like the idea of minimizing food waste.
\end{quotation}

\textbf{Interviewer:}
\begin{quotation}
    What should a food app contain / be able to do / help you with?
\end{quotation}

\textbf{Student 1:}
\begin{quotation}
    First of all provide me with a menu card - what’s on the menu - what can I get, the pricings, the opportunity to remove things from a dish if you have allergies or if you don’t like spicy stuff and payment methods.
\end{quotation}

\textbf{Interviewer:}
\begin{quotation}
    What do you think about an app that could show you where to find street food in Copenhagen?
\end{quotation}

\textbf{Student 1:}
\begin{quotation}
    Sure! Why not!? Yeah I like street food, I think it’s funny, I think it’s different!
\end{quotation}

\textbf{Interviewer:}
\begin{quotation}
    Have you ever had trouble finding a street food truck or?
\end{quotation}

\textbf{Student 1:}
\begin{quotation}
    No I think in Copenhagen, at least.. when there was ‘Papirøen’, it was in a hub and the same thing with Westmarket, everything is kinda in one place, so it’s not necessarily difficult to find, when it’s all gathered in places like that but I think if you start spreading it out and place trucks around the city, off course it will be difficult to locate them.
\end{quotation}

\textbf{Interviewer:}
\begin{quotation}
    So if there was such an app, you would rather use it for ordering food or?
\end{quotation}

\textbf{Student 1:}
\begin{quotation}
     Yeah or if I’m nearby then I would just go over there.
\end{quotation}

\textbf{Interviewer:}
\begin{quotation}
    What should/could this app be able to do if you should use it?
\end{quotation}

\textbf{Student 1:}
\begin{quotation}
    Well an overview of choices, a list, search engines, and a map, and everything I just described previously, would be an good idea, or maybe not actually.. I think kind off the charm with street food is that you go there and eat there, not that you order it and go home and eat, so I think actually maybe just create some sort of overview maybe a point system, and sometimes street food can be quite expensive.. maybe some sort of clip card system, where you get the tenth dish for free, I think that would be cool.
\end{quotation}

\textbf{Interviewer:}
\begin{quotation}
    I don’t have any more questions! Thank you for your time!
\end{quotation}


\subsection{Student 2:}

\textit{The interview person is informed about the project. - The recording starts after the first question is asked.}

\textbf{Interviewer:}
\begin{quotation}
    Do you use any apps to buy or to find food in Copenhagen?
\end{quotation}

\textbf{Student 2:}
\begin{quotation}
    Yes.
\end{quotation}

\textbf{Interviewer:}
\begin{quotation}
    What are they called?
\end{quotation}

\textbf{Student 2:}
\begin{quotation}
    Just Eat.
\end{quotation}

\textbf{Interviewer:}
\begin{quotation}
    Ok. And why do you use that? What do you like about that?
\end{quotation}

\textbf{Student 2:}
\begin{quotation}
    Because it is the one I know. It is international known. So when I came to Denmark, since I am an international. And since I came to Denmark I didn’t know any special ordering types. So I just looked up Just Eat, and that’s why. And yeah.
\end{quotation}

\textbf{Interviewer:}
\begin{quotation}
    What do you think then if we should create a food app, is there any features that would be nice to have, which is not necessarily on JustEat?
\end{quotation}

\textbf{Student 2:}
\begin{quotation}
    That it is available for every area. If there is different restaurants, the restaurants should specify which area or district in Copenhagen they can go to, before you go to pay. Because often it happens when you pay and you put in the address, and then they give you feedback that “Oh we don’t deliver to these places”. So something about that. And also think about the environment. Like if you do it on bike, or if you use a car. Like think about the environment, and use plastic bags when you deliver the food. Like Just Eat use plastic bags which is not like good.
\end{quotation}

\textbf{Interviewer:}
\begin{quotation}
    Then what do you think about an app that could show you other places, like street food vendors in Copenhagen? If you are in the city and looking for something, at it could show you on a map where you could find a specific street food vendor and the menu. Would you use that?
\end{quotation}

\textbf{Student 2:}
\begin{quotation}
    Yes. But I would also like to see the rating for that place.
\end{quotation}

\textbf{Interviewer:}
\begin{quotation}
    So you know if… ?
\end{quotation}

\textbf{Student 2:}
\begin{quotation}
    If I’m not gonna be sick. Like so you know that it is hygenic. When it is street food a lot of places doesn’t have water connected to the places, if it’s just in a car. Yeah so just ratings, so you could see the standard of that place. And also then it wouldn't be a fail,  if you go to a street food place that is on the app, and you come there and it is not what you expected. But if you have ratings, you can evaluate them.
\end{quotation}

\textbf{Interviewer:}
\begin{quotation}
    Ok. Perfect. Thank you.
\end{quotation}
